\chapter{Exhaustive Search}

Sometimes there seems to be no better approach than to examine every possible candidate in order to find one with a particular property or to show that one exists.
That, in essence, is exhaustive search.
Many of the algorithms we have met so far started life as exhaustive search algorithms.
By exploiting various \emph{monotonicity conditions} they were then transfomed into more efficient alternatives - \emph{greedy, thinning} or \emph{dynamic-programming} algorithms, algorithms whose running times were typically a low-order polynomial in the size of the input.

However, for many problems, even quite simply stated ones, no algorithm with a guaranteed polynomial running time is known.
For example, there is no known algorithm for determining the factors of a positive integer that takes polynomial time in the number of it digits.
We well explore a first example of such problems in the following section, known as the n-queens problem.

\section{A First Example: the n-Queens Puzzle}
The aim of the puzzle is to arrange \emph{n} queens on an $n \times n$ chessboard so that no queen attacks any other. Each queen therefore has to be placed on the board in a different row, column, and diagonal from any other queen.

The first two \emph{constraints} imply that any solution is necessarily a permutation of the numbers 1 to \emph{n} in which the \emph{j}th element is the number of the column in which the queen in row \emph{j} is placed.
For example, for the 8-queens problem there are 92 solutions, of which one is \texttt{1,5,8,6,3,7,2,4}. The queen in the first row is in column 1, the queen in the second row is in column 5, and so on.

The third constraint, having no two queens on the same diagonal, can be checked by exploiting the fact that in a grid diagonal cells are computed by adding or subtracting a constant factor from both the \emph{x-} and the \emph{y-coordinate} for any given cell.
Representing the position of a queen with its coordinates $(r, c)$ we can therefore define a safe position for another queen $(r', c')$ by ensuring that $abs(c' - c) \neq r' - r$.
An equivalent definition would be: $r + c \neq r' + c'$ (right diagonal) \emph{and} $r - c \neq r' - c '$ (left diagonal).

\subsection{A First Solution with Haskell}

\begin{impl}[Naive solution to the n-queens problem with Haskell]
We are implementing the diagonal safety condition by checking that the (absolute) difference of the x-coordinates of two queens is not equal to the difference of their y-coordinates, thus ensuring that we have no constant factor as described above.
Notice that we use `q' to denote the column of the queen, which makes the code more succinct.
Also note that the first two constraints of having no queens in the same row and column are ensured by only generating possible solutions from discrete permutations of the numbers 1 to n. 
\end{impl}

\begin{haskellcode}
safe :: [Int] -> Bool
safe qs = check (zip [1..] qs)

check :: (Eq a, Num a) => [(a, a)] -> Bool
check [] = True
check ((r,q):rqs) =
    and [abs (q' - q) /= r' - r | (r', q') <- rqs] && check rqs

queens :: Int -> [[Int]]
queens n = filter safe perms
    where perms = permutations [1..n]
\end{haskellcode}

Calling that function for the 4-queens problem yields:
\begin{verbatim}
ghci> queens 4
[[2,4,1,3],[3,1,4,2]]
\end{verbatim}
listing the only two possible configurations for placing 4 queens safely on a $4 \times 4$ chessboard.

Generating possible solutions from a permutation of the numbers 1 to \emph{n} does not only ensure the first two constraints but reduces also the number of possible solutions from $n^n$ to $n!$, even before any filtering or further checking is done. For $n=8$ this reduces the search space from 16,777,216 possibilities to 40,320 possible candidates, from which after the diagonal safety test only 92 remain as a valid solution.

But, after all, the running time of the naive algorithm is not good: generating the permutations takes $\Theta(n \times n!)$ steps and each safety test takes $\Theta(n^2)$ steps, so the full algorithm takes $\Theta(n^2 \times n!)$ steps because the safety test has to applied to $n!$ permutations. We can do better than that, and we will, but only after having a look at two other implementations in different languages and programming paradigms, namely logic programming in \emph{prolog} and imperative programming in \emph{java}.

\subsection{A First Solution with Prolog}

As we have seen above, this particular search problem can be solved by generating possible candidates and then checking if some given constraints do hold for any of the candidates.
In fact, algorithms of this kind are often called \textbf{constraint satisfaction} algorithms.
There are several programming environments available which are specialized in solving that kind of problems, providing algorithms for automatically checking if the given constraints are satisfied for a large number of variables, sometimes called \textbf{SAT-solvers}.

The programming language \emph{Prolog} provides such an environment, which - although not beeing a fully fledged SAT-solver - allows to formulate and check contraints in a concice and comprehensive manner without having to worry about implementing complex algorithms.
Prolog evaluates given logical statements and searches for solutions with a built-in three-step process:
\begin{itemize}
  \item \emph{unification}: the process of substituting variables with values according to the given facts and rules;
  \item \emph{backchaining}: if unification needs to evaluate further facts for compound rules, this is done automatically by chaining them together;
  \item \emph{backtracking}: if unification fails for the assigned values, prolog steps back recursively and assigns different matching values to the variables until a valid solution is found or the process finally fails.
\end{itemize}

We do not need to fully understand how prolog works in detail to digest the following example for the n-queens problem; it is sufficient to acknowledge that all the needed information is given by logical facts, rules, and constraints. Prolog will happily evaluate them within its three-step process when given a query and it will report the solution, if one exists.

\begin{impl}[Naive solution to the n-queens problem with Prolog]
With this implementation we are only solving the 4-queens problem, not because prolog couldn't handle bigger problems, but only because the choosen representation of facts would become somewhat tedious to maintain for many more variables.
So we kept it short and simple.
Observe, that we have a uniqueness constraint, ensuring that we have different values for all columns, which corresponds to the permutation constraint of the haskell version.
This also ensures the safety of queens in rows and columns, so we don't need extra constraints for that.
The diagonal safety constraints check against the positioning in diagonals.
Notice that they are defined in terms of `captures', so we need to negate them with {\mintinline{prolog}{\+}} in the main solution rule.
Also notice that there is no procedural logic included, no recursion and no backtracking as prolog will handle this automatically with its built-in process.
\end{impl}

\begin{prologcode}
% solve the 4-queens problem.
solution(C1,C2,C3,C4) :-
  uniq(C1,C2,C3,C4),
  \+ cap(2,C2,1,C1),
  \+ cap(3,C3,1,C1), \+ cap(3,C3,2,C2),
  \+ cap(4,C4,1,C1), \+ cap(4,C4,2,C2), \+ cap(4,C4,3,C3).

% columns
col(1). col(2). col(3). col(4).

% unique constraint
uniq(C1,C2,C3,C4) :- col(C1), col(C2), col(C3), col(C4),
    C1 \= C2, C1 \= C3, C1 \= C4, C2 \= C3, C2 \= C4, C3 \= C4.

% safety constraints
cap(R1,C1,R2,C2) :- R1-C1 =:= R2-C2.    % capture on left diagonal    
cap(R1,C1,R2,C2) :- R1+C1 =:= R2+C2.    % capture on right diagonal
\end{prologcode}

Evaluating that program in a prolog sessions yields the following results:
\begin{verbatim}
?- solution(C1,C2,C3,C4).
C1 = 2, C2 = 4, C3 = 1, C4 = 3 ;
C1 = 3, C2 = 1, C3 = 4, C4 = 2 ;
false.
\end{verbatim}
and we see that it finds the same two configurations as the haskell version.

\subsection{A First Solution with Java}

For solving this problem in an imperative style we use \emph{backtracking}, a recursive technique to generate solutions and (within the same step) to check if they are valid.
If they are not, the algorithm steps back from the generated candiadte and doesn't progress that solution any further.
Stepping back means that we are returning to the parent stack and choose a new value for this stage of evaluation.
This could lead to a situation in that, if we do not find a allowed configuration for a value in a solution, the algorithm steps back to the first value in the sequence and begins the search from scratch.
Backtracking is especially powerful when working with mutable data structures as arrays, in which we are repeatedly updating the links (pointers) to values of intermediate data structures, so that they eventually point to acceptable solutions.
This technique of \emph{dancing links} is described in full coverage in \autocite{taocp19}.

\begin{impl}[Backtracking with Java, taken from \autocite{introcs17}]
Again, we have a constraint check (method \texttt{isSafe}), ensuring that queens are placed correctly, and an additional pretty-print method for displaying a simple ascii graphic of the board.
The solution is recorded as an array of length n for the columns of the queens, having row numbers as indices as with the haskell solution.
The main work is done in the \texttt{generate} method, which generates in principle all possible combinations of rows and columns ($n^n$).
But with backtracking the constraint checking occurs during the generation of candidates (and not only after generating all of them), thus the exploration of further candidates is cut of in case of failing the safety test.
This leads to approxemately the same number of steps as our previous solutions.
\end{impl}

\begin{javacode}
public class Queens {
  public static boolean isSafe(int[] q, int n) {
    for (int i = 0; i < n; i++) {
      if (q[i] == q[n]) return false;         // same column
      if (q[i] - q[n] == n - i) return false; // same major diagonal
      if (q[n] - q[i] == n - i) return false; // same minor diagonal
    }
    return true;
  }

  public static void printQueens(int[] q) {
    int n = q.length;
    for (int i = 0; i < n; i++) {
      for (int j = 0; j < n; j++) {
        if (q[i] == j) System.out.print("Q ");
        else           System.out.print("* ");
      }
      System.out.println();
    }
    System.out.println();
  }

  public static void queens(int n) {
    int[] q = new int[n];
    generate(q, 0);
  }

  public static void generate(int[] q, int k) {
    int n = q.length;
    if (k == n) printQueens(q);
    else {
      for (int i = 0; i < n; i++) {
        q[k] = i;
        if (isSafe(q, k)) generate(q, k + 1);
      }
    }
  }

  public static void main(String[] args) {
    int n = Integer.parseInt(args[0]);
    queens(n);
  }
}
\end{javacode}

Calling queens(4) prints the generated solutions:
\begin{verbatim}
  * Q * * 
  * * * Q 
  Q * * * 
  * * Q * 
  
  * * Q * 
  Q * * * 
  * * * Q 
  * Q * *
\end{verbatim}
which are again identical, except that they are in reverse order, as well as the row numbers are.

\section{Running the Haskell Code}

\begin{haskellcode*}{gobble=2}

> import ExSearch
> main :: IO ()
> main = do
>   let qs = queens 8
>   print $ length qs
>   print qs

\end{haskellcode*}